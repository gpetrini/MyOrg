% Created 2022-09-13 ter 14:27
% Intended LaTeX compiler: pdflatex
\documentclass[11pt]{article}
\usepackage[utf8]{inputenc}
\usepackage{lmodern}
\usepackage[T1]{fontenc}
\usepackage[top=3cm, bottom=2cm, left=3cm, right=2cm]{geometry}
\usepackage{graphicx}
\usepackage{longtable}
\usepackage{float}
\usepackage{wrapfig}
\usepackage{rotating}
\usepackage[normalem]{ulem}
\usepackage{amsmath}
\usepackage{textcomp}
\usepackage{marvosym}
\usepackage{wasysym}
\usepackage{amssymb}
\usepackage{amsmath}
\usepackage[theorems, skins]{tcolorbox}
\usepackage[extrayear,uniquename=init,giveninits,justify,repeattitles,doi=false,isbn=false,url=true,maxcitenames=2,natbib=true,backend=biber]{biblatex}
\addbibresource{/home/gpetrini/Org/zotero_refs.bib}
\usepackage{url}
\usepackage[cache=false]{minted}
\usepackage[linktocpage,pdfstartview=FitH,colorlinks,
linkcolor=blue,anchorcolor=blue,
citecolor=blue,filecolor=blue,menucolor=blue,urlcolor=blue]{hyperref}
\usepackage{attachfile}
\usepackage{setspace}
\usepackage{tikz}
\author{Gabriel Petrini}
\date{2022}
\title{AB-SFC Ancona Summer School}
\begin{document}

\maketitle
:ID:       5d7e6b7b-e8e8-4c2b-918d-9aa3c5734aea

\section*{ABM: a tool for complex system - Gallegati}
\label{sec:org8762c28}

\begin{itemize}
\item ABM generates system changes endogenously
\begin{itemize}
\item Adaptative evolving complex system
\end{itemize}
\item Focus on interaction rather than optmizing behaviour
\item Emphasize opened and non-ergodic system
\end{itemize}


\section*{Introduction to R for ABM}
\label{sec:orgf54ab53}

\subsection*{Basic operations}
\label{sec:org839d058}

In R, \texttt{\%*\%} makes matriz multiplications.
Product elements, by the other hand, is generated with \texttt{*}.

\subsection*{Numerical solver}
\label{sec:org9e4a3ee}



\subsubsection*{Newton-type algorithm}
\label{sec:org6ffedb0}

In case of complex mathematical system, analytical solution is not possible.
The alternative is to rely on numeric solution algorithms.

The idea is that a very diffucult problem cam be solved approximating it with an easier one.
Instead of the hard to solve non-linear problem, we can solve an easier linear problem using Taylor expansion (\(g(x) = f(x_{k}) + f'(x_{k})(x - x_{k})\)).
Rearreaging, it is possible to find a proximation of \(f(x)\):

\[ x_{t+ 1} = x_{k} -  \frac{F(x_{k})}{F'(x_{k})}\]

This approach is limited, because the equation must be differentiable (at least once) and the derivative cannot be zero.
In addition, it is a \textbf{linear} approach (in the nearboorhood of evaluation).


\begin{enumerate}
\item Specify an innitual guess (\(x_{k} = x_{0}\))
\item Calculate the first step of the interation \(x_{k+ 1} = x_{k} - \frac{F(x_{k})}{F'(x_{k})}\)
\item Check if \(|F(x_{k+1})| \leq \delta\)
\begin{enumerate}
\item To find the sollution, it should interact until the value of the function is close to zero.
\end{enumerate}
\end{enumerate}

It is important to note that the initial guest is important.
If there are more than one solution, the Newton method will only find one at a time (and not globally).

\subsubsection*{The bisection method}
\label{sec:org42b0445}


\begin{enumerate}
\item specify initual values \(x_{lower} \leq x_{upper}\) such as \(f(x^{lower}) < f(x^{upper})\)
\item Compute the midpoint: \(x^{mid} = \frac{x^{lower} + x^{upper}}{2}\)
\item Update the bound:
\begin{enumerate}
\item If \(f(x^{mid}) \cdot f(x^{upper}) < 0\) than \(x^{upper} = x^{mid}\). Do not change \(x^{lower}\)
\item Otherwise \(x^{lower} = x^{mid}\). Do not change \(x^{upper}\)
\end{enumerate}
\item Calculate the value of the function is the midpoint: \(|f(x^{mid}| \leq \delta)\)
\item If step 4 is true, get the final results, otherwise go back to step 2
\end{enumerate}


\begin{verbatim}
[1] 2
\end{verbatim}

\subsubsection*{The secant method}
\label{sec:org872615e}

The ideia is that the \(f(x)\) is dificult to calculate.

\section*{SFC modelling theory - Zezza}
\label{sec:org1a7a1dc}

\subsection*{A first principle}
\label{sec:org466d982}

\begin{itemize}
\item Real wealth is accumulated over time with investment, and financial wealth (credit) with savings
\begin{itemize}
\item Must identify the sources of liquidity and expenditures and ensure that each payment from one sector is recorded as a receipts to other
\end{itemize}
\end{itemize}

The SFC approach have some similarities with the \href{monetary_circuit.org}{Monetary Circuit} approach, however there are some drawbacks.
For example, the latter does not includes capital goods payments or interest payment on loans.

\subsection*{Principles of SFC}
\label{sec:org08ea341}

\begin{enumerate}
\item Horizontal consistency
\item Vertical consistency
\item Flows-to-Stock consistency
\item Stock (balance sheets) consistency
\begin{enumerate}
\item The financial liabilities of an agent of sector are the financial assets of some other agent or sector
\item Net financial wealth for all sectores must be zero
\item Wealth for the world as a whole is only composed of real assets
\item Wealth for a single country is given by real wealth plus foreign assets, less foreign debt
\end{enumerate}
\item Stock-to-flows consistency
\begin{enumerate}
\item Stocks are accumulated for a purpose. So expenditure function must depend on accumulated wealth
\begin{enumerate}
\item Q: Does it implies the need to use New Cambridge equation?
\item Otherwise, the stock will grow indefinetly
\end{enumerate}
\end{enumerate}
\end{enumerate}

If the model has a steady growth, all stock and flows must grow at same ratio.
As a consequence, the ratios between variables will stabilize.
We can look at stock-flow ratios for a economy to check (in)stability.

\subsection*{SFC matrixes}
\label{sec:org6165169}


\subsubsection*{Transaction flow matrix}
\label{sec:org75fcd6e}

\begin{center}
\begin{tabular}{lllllll}
 & Households & Current Acc. (Firms) & Capital Acc. (Firms) & Banks & RoW & Total\\
\hline
\end{tabular}

\end{center}

\subsubsection*{Social Accounting Matrix}
\label{sec:orgb80d04b}

\begin{center}
\begin{tabular}{lcccccccc}
 & Production & Households & Non-financial firms & Financial firms & Government & RoW & Capital Accu. & Total\\
\hline
Production &  & \(C\) &  &  & \(G\) & \(E\) & \(I\) & \(Q\)\\
Households & \(W\) &  & \(TRfh\) & \(TRbh\) & \(TRgh\) & \(TRwh\) &  & \(Y_{h}\)\\
non-Financial firms & \(\Pi\) &  &  & \(TRbf\) & \(TRgf\) & \(TRwf\) &  & \(Y_{f}\)\\
Financial firms &  & \(TRhb\) & \(TRfb\) &  & \(TRgb\) & \(TRwb\) &  & \(Yb\)\\
Government & \(T\) & \(TRhg\) & \(TRfg\) & \(TRbg\) &  & \(TRwg\) &  & \(Y_{g}\)\\
RoW & \(M\) & \(TRhw\) & \(TRfw\) & \(TRbw\) & \(TRbw\) &  &  & \(Y_{w}\)\\
Capital Accu. &  & \(S_{h}\) & \(S_{f}\) & \(S_{b}\) & \(S_{g}\) & \(S_{w}\) &  & \(S\)\\
Total & \(Q\) & \(Y_{h}\) & \(Y_{f}\) & \(Y_{b}\) & \(Y_{g}\) & \(Y_{w}\) & \(I\) & \\
\end{tabular}

\end{center}

\subsubsection*{Flow of funds}
\label{sec:orgb4d5397}

\begin{center}
\begin{tabular}{lllllll}
 & Households & Non-financial firms & Financial firms & Government & RoW & Total\\
\hline
Real assets & \(I_h\) & \(I_f\) &  & \(I_{g}\) &  & \(+I\)\\
Deposits &  &  &  &  &  & \\
Loans &  &  &  &  &  & \\
Government Bill &  &  &  &  &  & \\
Equities &  &  &  &  &  & \\
Foreign debt &  &  &  &  &  & \\
Total & \(S_h\) & \(S_f\) & \(S_b\) & \(S_{g}\) & \(S_{w}\) & \(I\)\\
\end{tabular}

\end{center}

\subsection*{Flows to stock consistency}
\label{sec:orgc21ba7d}

End-of-period o level of period is the previous stock value (at constant prices)  plus a flow

\subsubsection*{Capital gains}
\label{sec:orgb033ad5}

At current prices, the change in sotck must take into account net capital gains.
Start from the previous identity and multiply by the price:

\[\Delta s_{t} = f_{t} + \Delta p\cdot s_{t-1}\]
Dividing by the price
\[\Delta S_{t} = F_{t} + \lambda\cdot S_{t-1}\]
The last term defines net capital gains.\footnote{Zezza points out that the dot-com-crisis was a capital gains-led cycle.}


\subsection*{Fundamental flows-to-stock links}
\label{sec:org40a34cf}

The stock of capital increases with net investment, for example:
\begin{itemize}
\item Household net financial assets increase with saving
\item Government debt increases with government borrowing
\item The net international position changes according to the current account balance
\end{itemize}

Note: According to Zezza, private sector have a positive balance due to uncertainty.
In this sense, government bills is one of the safest assets.

\subsection*{Closures}
\label{sec:orga7a1b53}

Once the SFC setup is complete, a nummber of model variables can be determined fom the identities implied by horizontal and vertical consistency.
The determination of the residual variables will depend on the choice of a specific thepry: this part of model development is refered to as a ``closure''.

\subsection*{Lab Exercise}
\label{sec:org181566e}


\subsubsection*{Recursive vs simultaneos models}
\label{sec:orga71fb0e}


The system:

\begin{align*}
 b_{11}y_{t} + b_{12}z_{t} & = f_{11}y_{t-1} + f_{12}z_{t-1} + \alpha_{1}\\
 b_{21}y_{t} + b_{22}z_{t} & = f_{21}y_{t-1} + f_{22}z_{t-1} + \alpha_{2}
\end{align*}
Can be written in matrix form as
\[BY_{t} = FY_{t-1} + A_{t}\]

The model is recursive if the B matrix is triangular.
In order to improve the chances to find a global solutions, modeling softwares rearrange model equations looking for triangular submatrices of B.

Normaly, software try to make matrix \(B\) triangular.
If it is not, implement interative methods.

\subsubsection*{Model's structure}
\label{sec:org222458c}


\begin{center}
\begin{tabular}{lccccccc}
\hline
 & Production & Workers & Capitalists & Firms (current) & Firms (Capital) & Banks & Total\\
\hline
Wages & \(-WB\) & \(+WB\) &  &  &  &  & 0\\
Net profits & \(-PR\) &  &  & \(+PR\) &  &  & 0\\
Depreciation & \(-DEP\) &  &  & \(+DEP\) &  &  & 0\\
\hline
GDP (income) & \([GDP]\) &  &  &  &  &  & \([GDP]\)\\
\hline
Interest on bonds &  &  & \(+i\cdot B_{c}\) & \(-i\cdot B\) &  & \(+i\cdot B_{b}\) & 0\\
Firms Dividends &  &  & \(+DIV_{f}\) & \(-DIV_{f}\) &  &  & 0\\
Bank Dividends &  &  & \(+DIV_{b}\) &  &  & \(-DIV_{b}\) & 0\\
\hline
Consumption & \(+C\) & \(C_{w}\) & \(C_{c}\) &  &  &  & 0\\
Investment & \(+I\) &  &  &  & \(-I\) &  & \\
Depreciation &  &  &  & \(-DEP\) & \(+DEP\) &  & \\
\hline
Change in bank deposits &  &  & \(-\Delta D\) &  &  & \(+\Delta D\) & \\
Change in bonds &  &  & \(-\Delta B_{c}\) & \(+\Delta B_{f}\) &  & \(-\Delta B_{b}\) & \\
\hline
Total & 0 & 0 & 0 & 0 & 0 & 0 & 0\\
\hline
\end{tabular}

\end{center}

Note: depreciation is repeated twice in the flow of funds matrix.

Assumptions:
\begin{itemize}
\item Assuming only bonds and money (banks deposits created by the banks)
\begin{itemize}
\item Firms finance investment by issuing bonds, demanded by banks and capitalists
\end{itemize}
\item Consumers do not save; so only capitalists are abble to buy bonds
\item All profits are distrubuted to the owners of the firms
\begin{itemize}
\item There is no equities for simplicity
\end{itemize}
\item Income distribution will be exogenous
\item 
\end{itemize}


\begin{center}
\begin{tabular}{lccccc}
\hline
 & Workers & Capitalists & Firms & Banks & Total\\
\hline
Real capital &  &  & \(+K\) &  & \(+K\)\\
Deposits &  & \(+D\) &  & \(-D\) & 0\\
Bonds &  & \(+B_{f} + B_{b}\) & \(-B_{f}\) & \(-B_b\) & 0\\
\hline
Total & 0 & \(V_{c}\) & 0 & 0 & \(+K\)\\
\hline
\end{tabular}

\end{center}

As a consequence of balance-sheet, capitalists hold all the stock of real capital by means of bank deposits and bonds.

\paragraph*{Identities}
\label{sec:org17f67db}

\begin{equation*}
\begin{align*}
 WB + PR + DEP & = Y + C + I\\
 C_{c}+ S_{c} & = iB_{c} + DIV_{f} + DIV_{b}\\
S_{c} & = \Delta D + \Delta B_{c}\\
DIV_{f} & = PR - iB\\
\Delta B & = I - DEP\\
DIV_{b} & = iB_{b}\\
\Delta D & = \Delta B_{b}
\end{align*}
\end{equation*}

\paragraph*{Equilibrium}
\label{sec:org68994e4}

\begin{description}
\item[{Good markets}] Investment-savings
\item[{Financial markets}] Bonds and bank deposits (hidden equation)
\end{description}

\subsubsection*{Code}
\label{sec:orge158ee7}
\paragraph*{Libraries}
\label{sec:orgbae0673}

\begin{minted}[frame=lines,fontsize=\scriptsize,linenos=]{r}
library(sfcr)
library(ggraph)
\end{minted}

Carregando pacotes exigidos: ggplot2

\paragraph*{Equations}
\label{sec:orgfa94973}

\begin{minted}[frame=lines,fontsize=\scriptsize,linenos=]{r}
eqs <- sfcr::sfcr_set(
               y ~ cons + i,
               pr ~  y - wb - dep,
               cw ~ wb,
               sc ~ ibc + divf + divb - cc,
               divf ~ pr - ibb,
               divb ~ ibb,
               yc ~ ibc + divf + divb,
               ibc ~ r*bc[-1],
               ibb ~ r*bb[-1],
               cons ~ cw + cc,
               wb ~ wshare * y,

               ## vc ~ d + bc,
               k ~ k[-1] + i  - depr*k[-1],
               dep ~ depr*k[-1],
               vc ~ vc[-1] + sc,

               cc ~ pc1*yc + pc2*vc[-1],
               b ~ b[-1] + i - dep,

               ## Portfolio decision
               bc ~ bc[-1] - bondssh*sc,
               bb ~ b - bc,
               d ~ bb,
               bondssh ~ 0.8

             )
\end{minted}

\paragraph*{Exogenous parameters}
\label{sec:org50837d9}

This is the initial calibration of the model.
To do so, it is possible to choose a scale value (in this case, GDP).


\begin{center}
\begin{tabular}{lccccccc}
\hline
 & Production & Workers & Capitalists & Firms (current) & Firms (Capital) & Banks & Total\\
\hline
Wages & -60 (endog) & \(+WB\) &  &  &  &  & 0\\
Net profits & -20 (residual) &  &  & \(+PR\) &  &  & 0\\
Depreciation & -20 &  &  & \(+DEP\) &  &  & 0\\
\hline
GDP (income) & 100 (exog) &  &  &  &  &  & \([GDP]\)\\
\hline
Interest on bonds &  &  & \(+i\cdot B_{c}\) & \(-i\cdot B\) &  & \(+i\cdot B_{b}\) & 0\\
Firms Dividends &  &  & \(+DIV_{f}\) & \(-DIV_{f}\) &  &  & 0\\
Bank Dividends &  &  & \(+DIV_{b}\) &  &  & \(-DIV_{b}\) & 0\\
\hline
Consumption & \(+C\) & \(C_{w}\) & \(C_{c}\) &  &  &  & 0\\
Investment & \(+I\) &  &  &  & \(-I\) &  & \\
Depreciation &  &  &  & \(-DEP\) & \(+DEP\) &  & \\
\hline
Change in bank deposits &  &  & \(-\Delta D\) &  &  & \(+\Delta D\) & \\
Change in bonds &  &  & \(-\Delta B_{c}\) & \(+\Delta B_{f}\) &  & \(-\Delta B_{b}\) & \\
\hline
Total & 0 & 0 & 0 & 0 & 0 & 0 & 0\\
\hline
\end{tabular}

\end{center}


\begin{center}
\begin{tabular}{lccccc}
\hline
 & Workers & Capitalists & Firms & Banks & Total\\
\hline
Real capital &  &  & 100 (exog) &  & \(+K\)\\
Deposits &  & 20 (residual) &  & -20 & 0\\
Bonds &  & 80 (assuming fix share) & -100 (endog) & 20 & 0\\
\hline
Total & 0 & 100 (endog) & 0 & 0 & \(+K\)\\
\hline
\end{tabular}

\end{center}

\begin{minted}[frame=lines,fontsize=\scriptsize,linenos=]{r}
external <- sfcr::sfcr_set(
                    y0 ~ 100,
                    depr ~ 0.2,
                    wshare ~ 0.6,
                    pc1 ~ 0.25,
                    pc2 ~ 0.15, ## adds up

                    # Exogenous
                    i ~ 20, ## Because of time constraints
                    r ~ 0.1
)


init <- sfcr::sfcr_set(
                    # Lagged values
                    k ~ 100,
                    vc ~ 100,
                    bc ~ 80,
                    b ~ 100,
                  )
\end{minted}

\paragraph*{Baseline solution}
\label{sec:org3f3f947}

\begin{minted}[frame=lines,fontsize=\scriptsize,linenos=]{r}
growth <- sfcr::sfcr_baseline(
  equations = eqs,
  external = external,
  periods = 10,
  method = "Broyden"
)
as.data.frame(sfcr_get_blocks(growth))
\end{minted}

   endogenous block
1     bondssh     1
2         dep     2
3           b     3
4           k     4
5         ibb     5
6         ibc     6
7        divb     7
8           y     8
9          pr     8
10         cw     8
11       divf     8
12         yc     8
13       cons     8
14         wb     8
15         cc     8
16         sc     9
17         bc    10
18         bb    11
19          d    12
20         vc    13

\href{uncertanty.org}{Uncertanty}
\end{document}
